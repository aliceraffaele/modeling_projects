Nonostante i vantaggi dell'algoritmo presentato, il codice pu� essere migliorato. Per esempio, si possono fare le seguenti modifiche:
\begin{itemize}
\item I dizionari (.keys all'interno del codice) potrebbero essere usati per i vertici e non per gli archi, ma in questo caso si dovrebbe risolvere il problema dei pesi da assegnare agli archi;
\item Gli esempi sono stati inseriti in un file python a parte e si potrebbe riuscire a strutturarlo come file di testo e richiamarlo nel file python matching;
\end{itemize}

L'idea iniziale era quella di utilizzare le funzioni Call-Back, cio� di utilizzarla all'interno del ciclo while come nel caso del TSP (Problema del commesso viaggiatore). In questo caso, tuttavia, le funzioni Call-Back non si potevano usare in quanto Gurobi non permette all'utente di utilizzare il valore delle variabili se non nella fase MIP, cio� in una fase del processo in cui si hanno solo valori interi delle soluzioni, ma il problema che si vuole aggirare con questo algoritmo � proprio quello di usare solo valori interi delle soluzioni.

Si � giunti quindi ad una contraddizione e si � dovuta scartare l'idea di utilizzare Call-Back all'interno del codice.

In conclusione, l'algoritmo pu� essere migliorato ed � stato aggirato il problema di utilizzare un certo tipo di funzioni all'interno del codice a causa di restrizioni del programma.