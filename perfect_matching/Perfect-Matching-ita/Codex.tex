\begin{verbatim}
@author: Alessandro Zeggiotti
# -------------------------------------------------------------------
# USEFUL MODULES

import math
import random
from gurobipy import *

import networkx as nx
import itertools

# Matching_examples.py contains some useful examples to test this script
# QUESTION: Is "import" the best way to run an external script?
from Matching_examples import *

# -------------------------------------------------------------------
# FUNCTIONS DEFINITION

# Given a graph, this function idefintifies a set of nodes into a single vertex

def Contraction(graph, nodes):
    V = set()
    for i,j in graph:
        V = V.union({i})
        V = V.union({j})
    new_vertex = max(V) + 1
    # Create a minor of the graph
    minor = {}
    for i,j in graph.keys():
        if i in nodes and j in nodes:
            pass
        elif i in nodes and (j,new_vertex) not in minor.keys():
            minor[j,new_vertex] = graph[i,j]
        elif i in nodes:
            minor[j,new_vertex] += graph[i,j]
        elif j in nodes and (i,new_vertex) not in minor.keys():
            minor[i,new_vertex] = graph[i,j]
        elif j in nodes:
            minor[i,new_vertex] += graph[i,j]
        else:
            minor[i,j] = graph[i,j]
    return minor

# Given a graph, this function finds the value of a minimum odd cut
# T is a datum used during the recursion: 
#it must be initially set equal to "vertices"

def minCutValue(graph, T):
    # Base step
    if T == set():
        return float('inf')
    # Create the graph using the module NETWORKX
    G = nx.DiGraph()
    for (i,j) in graph.keys():
        G.add_edge(i, j, capacity = graph[i,j])
        G.add_edge(j, i, capacity = graph[i,j])
    # Let s and t be two random nodes in T
    s = min(T)
    t = max(T)
    # NETWORKX contains a function that allows to find the minimum s-t cut
    # This function is based on Ford-Fulkerson algorithm
    cut_value, partition = nx.minimum_cut(G,s,t)
    S, S_bar = partition
    if len(S.intersection(T)) % 2 == 1: # if S is T-odd: ...
        return min(cut_value,
                   minCutValue(Contraction(graph, {s,t}), T - {s,t}))
    else:
        return min(minCutValue(Contraction(graph, S), T - S),
                   minCutValue(Contraction(graph, S_bar), T - S_bar))

# Given a set, this function finds all its subsets of a fixed cardinality

def findsubsets(S,m):
    # From ITERTOOLS module
    return set(itertools.combinations(S, m))

# Given a graph and the value of a minimum odd cut, 
# this function finds a minimum odd cut

def minCutEdges(graph, cut_value):
    for cardinality in range(1,n,2):
        subsets = findsubsets(vertices, cardinality) 
        # "vertices" is global variable
        for S in subsets:
            # Create the cut
            dS = {}
            for i,j in graph.keys():
                if i in S and j not in S:
                    dS[i,j] = graph[i,j]
                if i not in S and j in S:
                    dS[i,j] = graph[i,j]
            dS_value = sum(dS.values())
            if dS_value == cut_value:
                return dS
    print('No odd cut of value %g has been found' % cut_value)
    return None

# -------------------------------------------------------------------
# MODEL DEFINITION

m = Model()
# Suppress the standard output, it would be invoked too many times
m.setParam('OutputFlag', 0)

# Create the variables
vars = {}
for (i,j) in edges.keys():
   vars[i,j] = m.addVar(lb=0, vtype=GRB.CONTINUOUS, name='e[%d,%d]'%(i,j))

# To create the model data structure only once, after variables creation
m.update()

# Add the objective function
m.setObjective(sum(vars[i,j]*edges[i,j] for i,j in edges.keys()), GRB.MINIMIZE)

# Add degree-1 constraint
for i in vertices:
    # Look for the edges incident to i
    neighbors = []
    for j in vertices:
        if (i,j) in edges.keys():
            neighbors.append((i,j))
        elif (j,i) in edges.keys():
            neighbors.append((j,i))
    m.addConstr(sum(vars[i,j] for i,j in neighbors) == 1)

# -------------------------------------------------------------------
# MODEL OPTIMIZATION

linked = set()
for edge in edges.keys():
    linked = linked.union(set(edge))

# Check the presence of isolated vertices
if vertices != linked:
    print(str(vertices - linked) + ' is a set of isolated vertices!')

# Check the parity of the graph
elif n % 2 == 1:
    print('The graph is odd!')

else:
    # Now we are sure that the graph admits a solution
    while True:
        # Solve the problem
        m.optimize()
        print('\n--------------------------------------------------------\n')
        solution = m.getAttr('x', vars)
        selected = [(i,j) for i,j in solution.keys() if solution[i,j] != 0]
        print('New incumbent solution:\n')
        for i,j in selected:
            print('    ' + str((i,j)) + ' : ' + str(solution[i,j]))
        print('\nObjective function: %g' % m.objVal)
        print('Checking the presence of rationals...')
        # Find the value of a minimum odd cut
        cut_value = minCutValue(solution, vertices)
        if cut_value < 1:
            # Find the edges of a minimum odd cut
            cut_edges = minCutEdges(solution, cut_value)
            # Add a constraint
            m.addConstr(sum(vars[i,j] for (i,j) in cut_edges.keys()) >= 1)
            print('Minimum odd cut: %g' % cut_value)
            print('Related edges:\n')
            for i,j in cut_edges.keys():
                print('    ' + str((i,j)) + ' : ' + str(cut_edges[i,j]))
            print('\nNew constraint added...')
        else:
            print('This is a feasible solution!')
            break
\end{verbatim}